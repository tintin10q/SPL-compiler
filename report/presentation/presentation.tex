\documentclass{beamer}

\usepackage[export]{adjustbox}
\usepackage{amsmath,amsthm,amssymb,amsfonts}
\usepackage[backend=biber, style=apa]{biblatex}
\usepackage{caption}
\usepackage[T1]{fontenc}
\usepackage{graphicx}
\usepackage{lmodern}
\usepackage[outputdir=dist]{minted}
\usepackage{ru}
\usepackage{url}
\usepackage{xpatch,letltxmacro}
\usepackage{stmaryrd}
\usepackage{wrapfig}

\newcommand{\cd}[1]{\texttt{#1}}

\setminted{fontsize=\small}
\LetLtxMacro{\cminted}{\minted}
\let\endcminted\endminted
\xpretocmd{\cminted}{\RecustomVerbatimEnvironment{Verbatim}{BVerbatim}{}}{}{}

\mode<presentation>{}

\title{\textbf{Compiler Construction: Lexical analyses}}

\author{Quinten Cabo, Marijn van Wezel}

\institute{Radboud University Nijmegen}
\date{27 February 2024}

\begin{document}

\setbeamercovered{transparent}
\setbeamercolor{alerted text}{fg=black}
\setbeamerfont{alerted text}{series=\bfseries}

\frame{\titlepage}

\begin{frame}
  \frametitle{What was a compiler again?}
  
  \begin{center}
  \only<1>{\includegraphics[width=300px]{figures/compiler_phases.pdf}}
  \only<2>{\includegraphics[width=300px]{figures/compiler_phases_highlighted.pdf}}
  \end{center}
\end{frame}

\begin{frame}<1>[label=the-goal]
  \frametitle{The goal}

  \begin{itemize}[<+->]
    \item \alert<1>{Given some arbitrary string of characters,}
    \item \alert<2>{analyse this string,}
    \item \alert<3>{produce an abstract syntax tree,}
    \item \alert<4>{or give an error.}
  \end{itemize}
\end{frame}

\begin{frame}[fragile]
  \frametitle{String of characters: Examples}

  \begin{onlyenv}<1>
    \begin{minted}{text}
      sum(l:[Int]): Int {
        if (isEmpty(l.tl)) {
          return l.hd;
        }

        return l.hd + sum(l.tl);
      }
    \end{minted}
  \end{onlyenv}

  \begin{onlyenv}<2>
    \begin{minted}{text}
      product(list:[Int]): Int {
        if (isEmpty(list)) {
          return 1;
        }

        return list.hd * product(list.tl);
      }
    \end{minted}
  \end{onlyenv}

  \begin{onlyenv}<3>
    \begin{minted}{text}
      product(list:[Int]: Int {
        if (isEmpty(list)) {
          return 1;
        }

        return list.hd * product(list.tl);
      }
    \end{minted}
  \end{onlyenv}
\end{frame}

\againframe<2>{the-goal}

\begin{frame}
  \frametitle{Analysis: Haskell}

  We chose \textit{Haskell} to implement our parser (and the rest of our compiler):

  \begin{itemize}[<+->]
    \item Algebraic data types;
    \item High level of abstraction;
    \item Good tooling and library support;
    \item \textbf{Functional parser combinators}.
  \end{itemize}
\end{frame}

\begin{frame}
  \frametitle{Analysis: Functional parser combinators}

  Parser combinators work through \textit{function composition}.

  \begin{center}
    \includegraphics[width=300px]{figures/parser_combinator.pdf}
  \end{center}
\end{frame}

\begin{frame}
  \frametitle{Analysis: Megaparsec}

  \begin{figure}
    \begin{center}
      \includegraphics[width=0.7\textwidth]{figures/andromeda.jpg}
      \captionsetup{labelformat=empty}
      \caption{The Andromeda Galaxy is roughly one \textbf{megaparsec} away (Torben Hansen, CC-BY-2.0).}
    \end{center}
  \end{figure}
\end{frame}

\begin{frame}
  \frametitle{Analysis: Megaparsec}

  \begin{itemize}[<+->]
    \item Informal successor of Parsec;
    \item Does much of the heavy lifting;
    \item Large collection of combinators;
    \item Decent error messages out-of-the-box;
    \item \mintinline{text}{makeExprParser} 
  \end{itemize}
\end{frame}

\begin{frame}
  \frametitle{Analysis: Tokenless parsing}

  % whitespace
  % symbols
  % examples
  % megaparsec only removes whitespace at end
\end{frame}

\begin{frame}
  \frametitle{Analysis: Generating expression parsers}

  % field as expression
\end{frame}

\begin{frame}
  \frametitle{Analysis: Testing the parser}

  % hspec
\end{frame}

\againframe<3>{the-goal}

\begin{frame}
  \frametitle{Abstract syntax tree}

  % started with AST
\end{frame}

\againframe<4>{the-goal}

\begin{frame}
  \frametitle{Error reporting}
\end{frame}

\begin{frame}
  \frametitle{Pretty printer}

  % comments are hard
\end{frame}

\begin{frame}
  \begin{center}Questions?\end{center}
\end{frame}

\end{document}

